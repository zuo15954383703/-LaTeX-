\part{页面与页眉页脚}
\section{页面设置}
\begin{lstlisting}[style = LaTeX_TeXworks]
\documentclass[a4paper]{article}	% 设置纸张大小未A4纸张
\usepackage{geometry}				% 导入页面格式设置的代码
% 下面是两种设置页边距的方式
\geometry{left=2.5cm,right=2.5cm,top=2.5cm,bottom=2.5cm}
\geometry{margin=2.5cm}
\end{lstlisting}
\section{页眉页脚设置}
\begin{lstlisting}[style = LaTeX_TeXworks]
\usepackage{fancyhdr}				% 自定义设置页眉页脚的宏包
\footskip = 10pt					% 设置页脚边距
\renewcommand{\headrulewidth}{1pt}	% 设置页眉线的宽度
\renewcommand{\headwidth}{16cm}		% 设置页眉线的长度
\pagestyle{fancy}					% 将全局页眉页脚设置为fancy格式
\thispagestyle{empty}				% 删除该页的页眉页脚
\lhead{一份学习文档}					% 设置页眉左侧
\chead{nihao}						% 设置页眉中间
\rhead{我不知道写啥}					% 设置页眉右侧
\cfoot{\thepage}					% 设置页脚中间(\thepage表示当前页码)
\rfoot{}							% 设置页脚右侧
\lfoot{}							% 设置页脚左侧
\end{lstlisting}
值得注意的是,页眉页脚的格式在\LaTeX 中共存在四种模式:
\begin{enumerate}
	\item 
	empty:页面不显示页眉页脚。
	\item
	plain:页面显示页脚,但页眉为空。
	\item
	headings:页面显示页眉和页脚,页眉中包含章节标题或节标题。
	\item
	fancy:这是导入fancyhdr宏包后可以使用的格式,即自定义页眉页脚。
\end{enumerate}
