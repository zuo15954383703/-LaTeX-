\part{常见宏包介绍}
\LaTeX 实际上是以宏为基础的语言,所以严格而言,“包”在这里应当叫做宏包。宏以\addbs{def} 命令定义,此外,\addbs{newenvironment}、\addbs{newcommand} 和\addbs{renewcommand} 分别可以定义新环境、新宏和修改特定宏的作用。作为一篇简明教程,这里我们只进行“脚本小子”式的教学,讲述一些宏包的功能介绍,而略去关于宏本身的内容。
%\caption{常见宏包}\label{table::package_introducation}
\begin{center}
	\begin{longtable}{C{1.8cm}|L{13cm}}
		\hline
		宏 & 介绍\\
		\hline
		amsmath&提供了一系列功能强大的工具,用于编写复杂的数学公式和数学结构\\
		amsthm&提供了定义和排版定理、引理、证明等数学结构的功能\\
		ctex&为中文文档提供了完善的支持,包括中文字符的处理、中文文档类、章节标题格式等\\
		ifthen&提供了条件判断命令,可以在文档中根据不同条件执行不同的操作。\\
		titlesec&用于自定义和调整章节标题的格式和样式,包括字体、间距、对齐方式等。\\
		titletoc&用于定制目录的样式和格式,包括添加额外的目录内容、修改目录标题样式等。\\
		SIunits&提供了一套用于输入和排版科学单位的命令,以确保单位的一致性和标准化。\\
		tikz&强大的绘图工具,可以绘制各种类型的图形、图表和图示,支持高度自定义。\\
		extarrows&提供了一些额外的箭头符号,用于数学公式和图表中的指示和标记。\\
		indentfirst&自动缩进文档中每个段落的第一行。\\
		geometry&用于设置页面布局和页边距,可定制页面的大小、边距、页眉页脚等。\\
		multirow&提供了在表格中创建跨行单元格的功能。\\
		fancyhdr&用于自定义页面的页眉和页脚,可以添加页眉页脚内容和样式。\\
		lastpage&提供了一个命令,用于获取文档的总页数,方便在文档中引用。\\
		layout&显示当前页面布局的详细信息,包括页面尺寸、边距等。\\
		listings&用于排版代码清单,支持多种编程语言的语法高亮和格式设置。\\
		xcolor&用于设置文档中的颜色,支持各种颜色模型和色彩空间。\\
		multicol&用于创建多栏布局,可以在文档中同时显示多列内容。\\
		subcaption&用于支持子图和子表格,并提供了一些相关的命令和环境。\\
		graphicx&用于插入和处理图片,支持多种图片格式,并提供了一些图片调整和处理的命令\\
		algorithm2e&用于排版算法和伪代码,提供了一些用于排版算法的命令和环境\\
		dirtree&用于绘制目录结构的树状图,方便显示文件和文件夹的组织结构。\\
		menukeys&用于排版键盘快捷键和菜单,支持自定义快捷键的样式和格式。\\
		fontspec&提供了对字体的高级控制功能,支持使用系统安装的字体,并提供了一些字体设置命令。\\
		fontenc&用于指定字体编码,影响文档中字符的显示,常用于支持特定字符集。\\
		tipa&提供了一些国际音标的支持,用于排版国际音标符号。\\
		metalogo&用于排版各种 Logo,如 LaTeX、TeX 等。\\
		hyperref&用于创建超链接,可以在文档中添加链接到网页、章节、图片等的交互式链接。\\
		textgreek&提供了使用希腊字母的命令,方便在文档中输入希腊字母符号。\\
		chemfig&用于绘制化学结构式,支持排版复杂的化学分子结构。\\
		mhchem&用于排版化学方程式,提供了一套用于输入和排版化学式的命令和环境。\\
		array&提供了增强的表格功能,包括更灵活的列格式和表格样式设置。\\
		float&提供了对浮动对象(如图片和表格)的控制,可以设置浮动对象的位置和样式。\\
		circuitikz&用于绘制电路图,支持绘制各种类型的电路图元件。\\
		pgfplots&用于绘制高质量的图表,支持绘制二维和三维图表,并提供了丰富的绘图功能。\\
		nomencl&用于生成术语表,可以方便地生成和管理文档中使用的术语列表。\\
		glossaries&用于生成术语表和缩略词表,支持自动索引和排序术语,并提供了一些样式设置选项。\\
		longtable&提供了跨页的长表格功能,使得在文档中排版大型表格时更加灵活和方便。\\
		\hline
	\end{longtable}
\end{center}
