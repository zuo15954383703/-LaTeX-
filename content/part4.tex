\part{常见环境介绍}
\begin{lstlisting}[style = LaTeX_TeXworks]
% 正文区(唯一)
\begin{document}
\end{document}

% 图标区
\begin{figure}[H]
\end{figure}

% 表格区
\begin{table}[H]
\end{table}

% 标题页
\begin{titlepage}
\end{titlepage}

% 也是表格区
\begin{tabular}
\end{tabular}

% 长表格
\begin{longtable}
\end{longtable}

% 左对齐
\begin{flushleft}
\end{flushleft}

% 居中
\begin{center}
\end{center}

% 右对齐
\begin{flushright}
\end{flushright}

% 自定义区
\begin{<自定义的环境名称>}
\end{<自定义的环境名称>}

\end{lstlisting}

值得注意的是,环境是可以自己定义的,比如\addbs{def}和amsthm宏包中的\addbs{newtheorem}命令都可以定义新的环境。