\part{公式编辑}
\begin{lstlisting}[style = LaTeX_TeXworks]
\usepackage{amsthm}
\newtheoremstyle{Chinese_mathenv}{3pt}{3pt}{\KaiTi}{}{\SimHei \TimesNewRoman}{ }{.5em}{}
\theoremstyle{Chinese_mathenv}
\newtheorem{axiom}{公理}[section]
\newtheorem{theorem}{定理}[section]
\newtheorem{corollary}{推论}[axiom]
\newtheorem{definition}{定义}[section]
\newtheorem{lemma}{引理}[theorem]
\newtheorem*{caution}{注意}
\newtheorem*{solve}{解}
\newtheorem*{solution}{方案}
\end{lstlisting}
其他部分的编辑与Markdown极度相似,因此本文便不再赘述。详见\href{https://blog.csdn.net/weixin_46390192/article/details/133612181}{给自己看的Markdown语法大全}