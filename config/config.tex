\documentclass[12pt,a4paper]{article}
\usepackage{ctex}
\usepackage{float}
\usepackage{xcolor}
\usepackage{graphicx}
\usepackage{listings}
\usepackage{longtable}
\usepackage{subcaption}
\usepackage[colorlinks, hyperfootnotes = false]{hyperref}

\def\docTitle{一份\LaTeX 学习总结}

\usepackage{geometry}
\geometry{margin=2.5cm}

\usepackage{fancyhdr}
\renewcommand{\headrulewidth}{1pt}
\renewcommand{\headwidth}{16cm}
\pagestyle{fancy}
\chead{}
\rhead{\docTitle}
\lhead{\href{https://github.com/zuo15954383703/-LaTeX-}{https://github.com/zuo15954383703/-LaTeX-}}
\cfoot{\thepage}

\usepackage{indentfirst}
\setlength{\parindent}{2em}							% 设置段落的缩进为 2em,即段落开头缩进 2 个字符宽度。	
\linespread{1.2}	

\usepackage{fontspec}
\setCJKfamilyfont{Hei}{黑体}
\setCJKfamilyfont{Song}{宋体}
\setCJKfamilyfont{Kai}{楷体}
\newcommand{\Hei}{\CJKfamily{黑体}}
\newcommand{\Song}{\CJKfamily{宋体}}
\newcommand{\Kai}{\CJKfamily{楷体}}
\setCJKmainfont{宋体}

\newfontfamily{\TimesNewRoman}{Times New Roman}	% 导入本地Times New Roman字体
\setmainfont{Times New Roman}
\newcommand\resetfont{\Song \TimesNewRoman}	

\usepackage{amsmath,algorithm2e,subcaption,titlesec,listings,nomencl}
% 编号跟随:使得各类标题随着章节变动而重置计数器
\numberwithin{section}{part}
\numberwithin{footnote}{part}
\numberwithin{table}{part}
\numberwithin{figure}{part}
\numberwithin{algocf}{part}
% 编号样式和内容
\renewcommand{\thepart}{\Roman{part}}
\renewcommand{\thesection}{\arabic{part}.\arabic{section}}
\renewcommand{\thesubsection}{\thesection.\arabic{subsection}}
\renewcommand{\thefigure}{\arabic{part}.\arabic{figure}}
\renewcommand{\thesubfigure}{\alph{subfigure}}
\renewcommand{\thetable}{\arabic{part}.\arabic{table}}
\renewcommand{\thesubtable}{\alph{subtable}}
% 显示样式和内容
\titleformat{\part}{\centering \zihao{-3} \Hei \TimesNewRoman}{第 \thepart 部分}{1em}{}
\titleformat{\section}{\zihao{4} \Hei \TimesNewRoman}{\thesection}{1em}{}
\titleformat{\subsection}{\zihao{-4} \Hei \TimesNewRoman}{\thesubsection}{1em}{}
\titlespacing{\subsection}{2em}{*1}{*1}
\renewcommand{\tablename}{\Hei 表}
\renewcommand{\figurename}{\Hei 图}

\usepackage{titletoc}			
\setcounter{tocdepth}{2}
% \titlecontents{章节名称}[左端距离]{标题字体、与上文间距等}{标题序号}{空}{引导符和页码}[与下文间距]
\titlecontents{part}[0em]{\Hei \TimesNewRoman \vspace{5pt}}{\contentslabel{2em}}{}{~\titlerule*[0.6pc]{\textbullet}~\contentspage}
\titlecontents{section}[3em]{\Hei \TimesNewRoman}{\contentslabel{2em}}{}{~\titlerule*[0.6pc]{\textbullet}~\contentspage}
\titlecontents{subsection}[6em]{\Hei \TimesNewRoman}{\contentslabel{2em}}{}{~\titlerule*[0.6pc]{\textbullet}~\contentspage}

\usepackage{array,multirow,multicol}
\newcolumntype{L}[1]{>{\raggedright\let\newline\\\arraybackslash\hspace{0pt}}m{#1}}
\newcolumntype{C}[1]{>{\centering\let\newline\\\arraybackslash\hspace{0pt}}m{#1}}
\newcolumntype{R}[1]{>{\raggedleft\let\newline\\\arraybackslash\hspace{0pt}}m{#1}}

\newcommand\addbs[1]{\textbackslash #1} %% bs表示backslash,反斜杠
\newcommand\textlb{\textbackslash \textbackslash} %% lb表示line break,换行




